\chapter{Introduction}

\section{Revision History}

\begin{table}[htbp]
\begin{center}
\begin{tabular}[t]{cccc}

\hline
Version & Date & Author(s) & Summary\\
\hline
1.0 & 22/01/2017 & Francesco Larghi - Jacopo Fantin & First release\\
\hline
1.1 & 23/01/2017 & Francesco Larghi - Jacopo Fantin & Correction of problems\\
\hline

\end{tabular}
\end{center}
\end{table}

\section{Purpose and Scope}

This is the Project Plan Document (PPD) of our PowerEnJoy project system. 
\newline
The purpose of this document is to give an overall organization and estimation of the work from the documentation to the devolpment and the deployment of the system in order not to waste the available time and the man power.
This is useful to define a proper budget, time needed, resource allocation and the schedule of the activities.
These informations are absolutely significant and fundamental for the stakeholders.
\newline
In the first part we will use Function Points and COCOMO techniques to estimate the expected size of our project in terms of time and lines of code and the cost/effort required for the development.
\newline
In the second part we will propose an idea for the schedule of all the project phases and activities.
\newline
Then we will propose possible assignments of resources (members of our group) for each identified task.
Finally, we will focus on possible risks of the project and general conclusions.


\section{List of Definitions and Abbreviations}

\begin{itemize}
\item \textbf{PPD:} Project Plan Document.
\item \textbf{RASD:} Reqirements Analysis Specifications Document.
\item \textbf{DD:} Design Document.
\item \textbf{DB:} DataBase.
\item \textbf{DBMS:} DataBase Management System.
\item \textbf{Component:} Software element that implements functionalities.
\item \textbf{FP:} Function Points.
\item \textbf{ILF:} Internal Logic File.
\item \textbf{EIF:} External Interface File.
\item \textbf{EI:} External Inputs.
\item \textbf{EO:} External Outputs.
\item \textbf{EQ:} External Inquiries.

\end{itemize}

\section{List of Reference Documents}

\begin{itemize}
\item[\textbf{--}] Our PowerEnjoy Requirements Analysis Specifications Document
\item[\textbf{--}] Our PowerEnjoy Design Document
\item[\textbf{--}] Our PowerEnjoy Integration Testing Plan Document
\item[\textbf{--}] The specification document: Assignments AA'16-'17.pdf
\item[\textbf{--}] Examples of Project Plan Document available
\end{itemize}